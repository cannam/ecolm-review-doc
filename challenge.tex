
% Software Sustainability Challenge submission should be no more than
% 2-pages long, and use the ACM sigconf template (for more details,
% see the main conference Call For Papers). They will be peer
% reviewed, and accepted submissions will be presented at the
% conference as either part of a panel or lightning talk and/or within
% a poster session (as determined by the Programme Chair). Challenge
% submissions will not be included in the main DLfM proceedings, but
% will be published on the conference website; they will additionally
% be used (with appropriate credit given) to inform a report into
% Digital Musicology for the UK Software Sustainability Institute.
%
% Challenge contributions must be submitted via the DLfM 2023
% EasyChair page by Friday 29 September.

\documentclass[sigconf]{acmart}
%\documentclass[sigconf, nonacm=true, authordraft=true, anonymous=true]{acmart}
\setcopyright{none}

\begin{document}
\title{Software Sustainability Challenge: ECOLM and Lute Tablature}

\acmConference[DLfM'23]{10th International Conference on Digital Libraries for Musicology}{November 10, 2023}{Milan, Italy}

\author{Chris Cannam}
\orcid{https://orcid.org/0009-0001-9814-6512}
\affiliation{%
  \institution{Particular Programs Ltd}
  \city{London}
  \country{UK}}
\email{chris.cannam@particularprograms.co.uk}

\author{David Lewis}
\affiliation{%
  \institution{Goldsmiths, University of London}
  \city{London}
  \country{UK}}
\email{d.lewis@gold.ac.uk}

\author{Tim Crawford}
\affiliation{%
  \institution{Goldsmiths, University of London}
  \city{London}
  \country{UK}}
\email{t.crawford@gold.ac.uk}

\maketitle
\begin{sloppypar}

  % Authors should use headings from the following list to structure
  % their submissions, addressing implications for sustainability in all
  % relevant sections, and using as many as are applicable for their
  % scenario:
  %
  % ·       Nature and purpose of the sustainable tool or resource
  % - Yes
  %
  % ·       Audience and users, disciplines and subjects
  % - Yes
  %
  % ·       Position within the research lifecycle
  % - Possibly not
  %
  % ·       Means of access and accessibility
  % - Possibly not
  %
  % ·       Current and future requirements, or implementations
  % - Yes
  %
  % ·       Challenges for sustainability
  % - Yes, especially
  %
  % ·       Future directions
  % - Yes, tentatively

  \section{Nature and purpose of the sustainable tool or resource}

  {\bf ECOLM} (1999-2002) was a project run by Tim Crawford which
  developed and populated a queryable database of lute tablature
  encodings with metadata using a web interface. Subsequent projects
  {\bf ECOLM II} (2002-2006) and {\bf ECOLM III} (2012) expanded the
  database and used it for some computational musicological
  investigations. The resulting database was hosted on a public-facing
  web server at Goldsmiths, University of
  London.\footnote{http://doc.gold.ac.uk/isms/ecolm/database/} It is
  still running today, although nobody is formally responsible for
  maintaining it.

  ECOLM is a relatively small scholarly resource with around 2,000
  carefully-curated encodings. A number of other public lute tablature
  resources exist, of variable size, quality, and consistency (see
  section \ref{audience}). These typically face challenges to
  sustainability similar to those of ECOLM (see section
  \ref{challenges}). It seems reasonable to try to address those
  challenges for such resources in general, at least insofar as they
  are useful to the same audience. Some of the maintainers of other
  resources have offered to contribute their data to such an effort.
  
  \section{Audience and users, disciplines and subjects}\label{audience}
  
  \section{Challenges for sustainability}\label{challenges}

  \section{Future directions}

  We have identified three alternative directions for sustainable
  development.
  
  \subsection{``Enhanced ECOLM''}

  This approach retains the relational data schema of the existing
  ECOLM, which is detailed and fairly effective, although it has
  little in common with the other systems we have considered or with
  wider current practice. We then provide ETL-type (extract,
  transform, load) data loaders for other sources of interest.

  Advantages of this approach include the ability to preserve existing
  code and to use original ECOLM records as a reference. The existing
  schema provides appropriate structure and reflects some good
  domain-specific decisions. Relational data import is a well
  understood field, and we could focus on user interfaces and data
  conversion rather than any novelty of data representation.

  Disadvantages include that the schema has little in common with any
  of the ad-hoc solutions other maintainers have settled on, so all
  import and export would be custom. The schema is perhaps already
  overspecified for its current use, yet does not address any problems
  relating to stable identifiers, versioning, or providing queryable
  APIs or data sources.

  Although we could at least initially reuse the existing user
  interface, it is no longer considered a strength of the system and
  would need some work to update to modern expectations.

  \subsection{Graph-based}

  In this approach we take the fundamental representation to be a
  graph of triples in the model of RDF, and convert all metadata to
  that for import and from it for query. External data such as
  transcriptions and multimedia resources are identified by
  graph-relatable identifiers such as URIs.

  Advantages include the use of a widely-understood and accepted model
  that meets common expectations about data compatibility and API
  provision. For schema we can draw ontologies from a number of
  existing systems. The structure is reasonably amenable to versioning
  and to use of ``idempotent'' import flows with automated testing,
  offering the option of ongoing import of changes in upstream
  sources. In principle existing tools may be used for review, query,
  inferencing, and format conversion.

  The approach has difficulties as well. It discards the existing user
  interface work and requires even the existing ECOLM data to be
  converted. Although graph representations have wide application,
  they are not generally used for manual data management and therefore
  have as little in common with the ad-hoc schemas of enthusiast lute
  resources as with that of ECOLM. Significant work would be required
  to maintain stable identifier mappings from external sources. With a
  more flexible structure than ECOLM's relational database, care and
  good automated testing would be needed to avoid ``silently missing
  data'' problems on query. Finally a separate solution would be
  needed to the problem of identifying and retrieving non-graph data
  such as media resources.

  Although in this approach we could no longer use the existing ECOLM
  user interface, that may be slightly mitigated by the ability to
  adapt other graph-driven UIs to the model.
  
  \subsection{``RISM-aligned''}

  Fundamental metadata representation is MARC, and we use Muscat to
  maintain it? 
  
\end{sloppypar}
\end{document}
