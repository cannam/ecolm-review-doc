
\documentclass[twocolumn,10pt]{paper}
\usepackage[a4paper,margin=0.8in]{geometry}

\usepackage[T1]{fontenc}
\usepackage{tgschola}

\begin{document}
\title{ECOLM Futures Review}

\maketitle
\begin{sloppypar}
  
  %\begin{abstract}
  %\end{abstract}
  
  {\bf Status: Preliminary thought-dump}
  
  \section{Background and Motivation}

  \subsection{What is ECOLM?}

  ECOLM\footnote{http://igor.gold.ac.uk/isms/ecolm/}, or ``Electronic
  Corpus of Lute Music'', is the name of a series of research projects
  aiming to develop a queryable online database of lute tablature
  encodings, of quality suitable for scholarly use.

  Two critical design goals are:
  
  \begin{enumerate}
  \item It gives encodings of music, not only metadata;
  \item It is trustworthy for scholarly use: for example, sources are
    identified, reliability of attribution is noted, editorial changes
    are pointed out, and it distinguishes between performance and
    diplomatic transcriptions.
  \end{enumerate}
  
  Here we use the name ECOLM broadly to refer to this design of
  database application, as well as to the past research projects with
  that name and to the existing
  system\footnote{http://doc.gold.ac.uk/isms/ecolm/database/} that
  they produced.

  \subsection{Aim of this Review}

  We aim to review the premise and outcomes of ECOLM and to consider
  whether a ``lightweight path to sustainability'' can be found that
  can be incrementally extended to other tablature resources.

  The following paragraphs set out the history of ECOLM and enumerate
  other resources of interest.

  \subsection{ECOLM History}

  %%%!!! + bibliography
  %%%!!! + URLs in footnote
  \begin{itemize}
  \item {\bf ECOLM} (1999-2002) was a project run by Tim Crawford
    which produced a queryable database of lute encodings with
    metadata with a web interface. Presented through a public-facing
    server at Goldsmiths, the service is still accessible today.
  \item {\bf ECOLM II} (2002-2006) was a successor project which
    expanded the ECOLM database and used it for some computational
    musicological investigations.
  \item {\bf ECOLM III} (2012) was a short project with the goal of
    adding further high-quality encodings by crowd-sourcing
    corrections of OMR (optical music recognition) scans.
    %%%!!! say more about this later
  \end{itemize}

  The ECOLM database as available online contains about 2,000
  tablature encodings, manually curated, of relatively high quality
  with accompanying metadata.

  \subsection{Other Lute Tablature Resources}

  Several other collections of lute music have been placed online by
  various curators. Some of these are licensed appropriately for
  inclusion in any future dataset, or have been offered for this use.
  In some cases there are concerns about their ongoing sustainability
  that make it particularly desirable to include them in future work.

  \begin{itemize}
    \item {\bf Lutemusic.org}\footnote{https://lutemusic.org/} curated
      by Sarge Gerbode. Around 20,000 encodings in playing editions
      with semi-structured metadata, informally curated with limited
      version tracking or editorial notes.
    \item {\bf Mss.slweiss.de}\footnote{https://mss.slweiss.de/}
      curated by Peter Steur and the late Markus Lutz. A metadata
      catalogue of around 68,000 listings of which the majority have
      incipits (opening ideas) encoded.
    \item {\bf Lute Society publications} curated by John
      Robinson. Scans from printed periodicals intended for players,
      containing around 7,000 encodings consisting of printed music,
      prose commentary, and semi-structured metadata.
    \item {\bf Phal\`ese} curated by Jan Burgers. Around 1,000
      encodings transcribed from editions of 16th-century publisher
      Pierre Phal\`ese with publication metadata.

      %%!!! tabulate their type, content, whether published already,
      %% licence terms, how widely used
  \end{itemize}

  %%!!! include section about other non-lute resources: RISM being the
  %%!!! most obvious but also anything else anyone else mentioned
  
  \section{User Context}

  We conducted informal interviews with three exemplary users of
  online early-music resources, in order to understand scholarly
  expectations. These were a ``traditional'' musicologist, a
  computational musicologist, and a lute performer and teacher.
  
  \subsection{Musicologist}

  The musicologist gave the following indications about their use of
  online resources of this type:

  \begin{itemize}
    \item Depending on the material, may begin by searching
      RISM\footnote{https://rism.info/} or
      Cantus\footnote{https://cantus.uwaterloo.ca/} databases;
    \item Routinely starts with a search by composer or source, since
      titles tend to have too many historical variants;
    \item Finds diplomatic transcriptions (i.e. closely following
      the source without editorial intervention) the most useful,
      but grateful for any transcription;
    \item Always refers to the facsimile as well, regardless of status
      of transcriptions, so can often do without editorial notes;
    \item Is particularly interested in dual tablature and staff
      renderings;
    \item Would appreciate opportunity to annotate or correct
      unreliable transcriptions.
  \end{itemize}
  
  \subsection{Computational Musicologist}

  The computational musicologist gave the following indications about
  their use of such resources:

  \begin{itemize}
  \item Will typically begin by searching RISM, and trusts that
    metadata in RISM is more authoritative than elsewhere;
  \item Finds trust very important, appreciating annotations about the
    original source, transcriber, and editorial interventions;
  \item Can work with unreliable transcriptions if their quality is
    known and original sources are properly described;
  \item Appreciates a simple presentation and single search function
    as their first entry point;
  \item Finds the ability to refine results via facets more useful
    than the ability to construct complex queries from the outset;
  \item Wants the ability to download results (up to the whole
    dataset) or query via API, to use with computational tools such
    as music21 or Humdrum locally.
  \end{itemize}
  
  \subsection{Lute Performer and Teacher}

  The lute performer and teacher gave the following indications about
  their use:

  \begin{itemize}
    \item Will often begin using the most informal performance
      resources because they have the most material, but this causes
      problems cross-referencing with more authoritative material;
    \item Finds information about the original source, transcriber,
      editorial interventions extremely important;
    \item Expects students to know those things about the editions
      they use when performing;
    \item Would greatly appreciate something containing modern
      performing editions as at lutemusic.org but with more reliable
      editorial commentary;
    \item In the absence of trustworthy information about the
      transcription, needs to compare every note with facsimile before
      using.
  \end{itemize}

  \subsection{Common Threads}

  Trust and provenance are common themes in discussion with all three
  of our exemplary users. They have different requirements for
  content, format, detail of editorial notes and so on, but share a
  desire to know the quality of transcription and level of editorial
  intervention they are dealing with.

  The musicological specialists were comfortable with RISM and would
  prefer some level of compatibility, perhaps as far as having the
  works indexed from RISM and metadata managed there.

  None of the three indicated they would hope to {\em contribute}
  material to a dataset like this, although they might appreciate the
  ability to make corrections.
  
  \section{Technical Review}

  \subsection{Tablature Resources}
  
  \subsubsection{ECOLM}

  \begin{itemize}
  \item SQL database
  \item Entity relationships modelled in schema---pre-RDF, not
    triples, relations hardcoded
  \item Structured using ``clusters'' (give examples)---attempt to
    support more general relations within schema
  \item Confidence levels modelled
  \item Same database used for user logins / editorial control as for
    content records---expectation that data managed ``within ECOLM''
  \item Degree of rigour in organisation means that, while it may be
    tricky to convert or adapt to another format or system, such an
    effort will probably succeed without too many loose ends
  \end{itemize}
  
  \subsubsection{lutemusic.org}

  \begin{itemize}
  \item Hierarchical organisation with separate trees by composer,
    source, and facsimile
  \item Composer and source trees contain Fronimo tab transcriptions
    with derived MIDI and PDF renderings
  \item Facsimile tree contains images (typically PNG) closely cropped
    with thresholding, apparently intended for reading from screen
    rather than as historical page facsimiles
  \item Separate tab hierarchy also present with tab-format files,
    probably older
  \item Hand-maintained spreadsheet contains metadata and index for
    Fronimo files
  \item Website presents the filesystem hierarchy directly (entirely
    static, uses web server index pages, nothing generated)
  \item Irregular organisation may make adaptation relatively high
    risk
  \end{itemize}

  \subsubsection{mss.slweiss.de}

  \begin{itemize}
  \item PHP web application driven entirely from CSV files (actually
    semicolon-separated rather than comma-separated)
  \item Flat directory containing one CSV file per source
  \item Incipits embedded in the CSV files, in ABC format, rendered to
    SVG from the PHP scripts for serving to browser
  \item Separate index CSV files list manuscript metadata and
    concordances
  \item Version-controlled since 2013
  \item Organisation seems tidy and easy to deal with
  \end{itemize}
    
  \subsubsection{Lute Society}

  \begin{itemize}
  \item Lute News facsimiles and transcriptions organised by issue
  \item Organisation is on filesystem, with PDFs and transcriptions of
    both text and tablature
  \item Simple front-end added by Tim
    Crawford\footnote{https://doc.gold.ac.uk/~mas01tc/jhr\_web/}
    provides a web index via Javascript requests from client
  \item Seems well arranged, the biggest problem looks like the wide
    variety of types of material present and the original linear
    organisation for readers and players
  \end{itemize}
  
  \subsubsection{Phal\`ese}

  \subsection{Other Related Sites}
  
  \subsubsection{RISM}
  \subsubsection{Vihuela Database}
  \subsubsection{Josquin Research Project}
  \subsubsection{DIAMM}
  \subsubsection{earlymusicsources.com}

  \section{Desirable Qualities of a Solution}
  \subsection{Social}

  \begin{itemize}
  \item talk about types of user and their expectations
  \item talk about crowdsourcing and ECOLM III
  \end{itemize}
  
  \subsection{Technical}
  \subsection{User Experience}

  \section{Proposed Path}

  
\end{sloppypar}
\end{document}
