
% Software Sustainability Challenge submission should be no more than
% 2-pages long, and use the ACM sigconf template (for more details,
% see the main conference Call For Papers). They will be peer
% reviewed, and accepted submissions will be presented at the
% conference as either part of a panel or lightning talk and/or within
% a poster session (as determined by the Programme Chair). Challenge
% submissions will not be included in the main DLfM proceedings, but
% will be published on the conference website; they will additionally
% be used (with appropriate credit given) to inform a report into
% Digital Musicology for the UK Software Sustainability Institute.
%  
% Authors should use headings from the following list to structure
% their submissions, addressing implications for sustainability in all
% relevant sections, and using as many as are applicable for their
% scenario:
%
% ·       Nature and purpose of the sustainable tool or resource
% ·       Audience and users, disciplines and subjects
% ·       Position within the research lifecycle
% ·       Means of access and accessibility
% ·       Current and future requirements, or implementations
% ·       Challenges for sustainability
% ·       Future directions
%
% Challenge contributions must be submitted via the DLfM 2023
% EasyChair page by Friday 29 September.

\documentclass[sigconf]{acmart}
%\documentclass[sigconf, nonacm=true, authordraft=true, anonymous=true]{acmart}
\setcopyright{none}

\begin{document}
\title{Software Sustainability Challenge: ECOLM and Lute Tablature}

\acmConference[DLfM'23]{10th International Conference on Digital Libraries for Musicology}{November 10, 2023}{Milan, Italy}

\author{Chris Cannam}
\orcid{https://orcid.org/0009-0001-9814-6512}
\affiliation{%
  \institution{Particular Programs Ltd}
  \city{London}
  \country{UK}}
\email{chris.cannam@particularprograms.co.uk}

\author{David Lewis}
\affiliation{%
  \institution{Goldsmiths, University of London}
  \city{London}
  \country{UK}}
\email{d.lewis@gold.ac.uk}

\author{Tim Crawford}
\affiliation{%
  \institution{Goldsmiths, University of London}
  \city{London}
  \country{UK}}
\email{t.crawford@gold.ac.uk}

\maketitle
\begin{sloppypar}


  
  
\end{sloppypar}
\end{document}
